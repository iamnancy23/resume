%%%%%%%%%%%%%%%%%
% This is an sample CV template created using altacv.cls
% (v1.1.4, 27 July 2018) written by LianTze Lim (liantze@gmail.com). Now compiles with pdfLaTeX, XeLaTeX and LuaLaTeX.
% 
%% It may be distributed and/or modified under the
%% conditions of the LaTeX Project Public License, either version 1.3
%% of this license or (at your option) any later version.
%% The latest version of this license is in
%%    http://www.latex-project.org/lppl.txt
%% and version 1.3 or later is part of all distributions of LaTeX
%% version 2003/12/01 or later.
%%%%%%%%%%%%%%%%

%% If you need to pass whatever options to xcolor
\PassOptionsToPackage{dvipsnames}{xcolor}

%% If you are using \orcid or academicons
%% icons, make sure you have the academicons 
%% option here, and compile with XeLaTeX
%% or LuaLaTeX.
% \documentclass[10pt,a4paper,academicons]{altacv}

%% Use the "normalphoto" option if you want a normal photo instead of cropped to a circle
% \documentclass[10pt,a4paper,normalphoto]{altacv}

\documentclass[12pt,a4paper]{altacv}
%% AltaCV uses the fontawesome and academicon fonts
%% and packages. 
%% See texdoc.net/pkg/fontawecome and http://texdoc.net/pkg/academicons for full list of symbols.
%% 
%% Compile with LuaLaTeX for best results. If you
%% want to use XeLaTeX, you may need to install
%% Academicons.ttf in your operating system's font 
%% folder.


% Change the page layout if you need to
\geometry{left=1.25cm,right=10cm,marginparwidth=8cm,marginparsep=1cm,top=0.70cm,bottom=1.00cm}

% Change the font if you want to.

% If using pdflatex:
\usepackage[T1]{fontenc}
\usepackage[utf8]{inputenc}
\usepackage[default]{lato}

% If using xelatex or lualatex:
% \setmainfont{Lato}

% Change the colours if you want to
\definecolor{LightGreen}{HTML}{1B4F72}
\definecolor{Lightgr}{HTML}{012AC2}
\definecolor{Mulberry}{HTML}{72243D}
\definecolor{SlateGrey}{HTML}{2E2E2E}
\colorlet{heading}{LightGreen}
\colorlet{accent}{Lightgr}
\colorlet{emphasis}{SlateGrey}
\colorlet{body}{black}

% Change the bullets for itemize and rating marker
% for \cvskill if you want to
\renewcommand{\itemmarker}{{\small\textbullet}}
\renewcommand{\ratingmarker}{\faCircle}
%% sample.bib contains your publications
\addbibresource{sample.bib}

\usepackage[colorlinks]{hyperref}

\begin{document}

\name{Aseem Ranjan}
\tagline{ }
\personalinfo{%
  % Not all of these are required!
  % You can add your own with \printinfo{symbol}{detail}
  \email{\href{mailto:rathorenav123@gmail.com}{\textcolor{black}{aseemranjan10@gmail.com}}}
  \phone{\href{tel:+917726063090}{\textcolor{black}{+91 9982054542}}}
  \linkedin{\href{https://www.linkedin.com/in/aseemranjan/}{\textcolor{black}{aseemranjan}}}
  
\hspace{1cm}

%   \github{\href{https://www.github.com/krishrustagi/}{\textcolor{black}{krishrustagi}}}
%   \link{\href{https://www.linkedin.com/in/krish-rustagi/}{\textcolor{black}{Codechef: krishrustagi}}}
%   \link{\href{https://www.github.com/krishrustagi/}{\textcolor{black}{Codeforces: mr\_cruise}}}
%   \link{\href{https://www.github.com/krishrustagi/}{\textcolor{black}{Codeforces: mr\_cruise}}}
  

  %% You MUST add the academicons option to \documentclass, then compile with LuaLaTeX or XeLaTeX, if you want to use \orcid or other academicons commands.
%   \orcid{orcid.org/0000-0000-0000-0000}
}

%% Make the header extend all the way to the right, if you want. 
\begin{fullwidth}
\makecvheader
\end{fullwidth}

%% Depending on your tastes, you may want to make fonts of itemize environments slightly smaller
% \AtBeginEnvironment{itemize}{\small}


%% Provide the file name containing the sidebar contents as an optional parameter to \cvsection.
%% You can always just use \marginpar{...} if you do
%% not need to align the top of the contents to any
%% \cvsection title in the "main" bar.

%----------------------------------------------------------------------------------------
%	WORK EXPERIENCE SECTION
%----------------------------------------------------------------------------------------
% \cvsection[page1sidebar]{EXPERIENCE}

% \cvevent{Android Developer Intern}{Spacenos}{Mar 2021 - Apr 2021}{}
% \begin{itemize}
% \item Worked on the Front-end of the Android App (Tradies On Demand).
% \item \textbf{Integrated the Firebase} to store our services, user details, profile images,
% licences of tradies and user authentication.
% \item The app is to connect customers who need services directly with \textbf{local licenced
% tradies}.
% \end{itemize}



%----------------------------------------------------------------------------------------
%	Experience SECTION
%----------------------------------------------------------------------------------------

\cvsection[page1sidebar]{\Large Experience}
\cvevent{\large Software Engineer Intern \hspace{0.3em}}{Eigenmaps.ai}{05/2022 - 11/2022}{}
\begin{itemize}
\item{\small Contributing to the backend team involving data handling and ensuring that live projects were executed successfully.}
\item {\small Responsible for troubleshooting and issue analysis, as well as coding, testing and implementing software enhancements.}
\item { \small Worked on developing a \textbf{dynamic threshold} to detect real-time anomalies in the univariate metric.}
\item { \small Build the \textbf{real-time logger} for fetching logs from the API for model performance and error analysis.}
\item { \small Developed and implemented efficient \textbf{APIs}, improving system performance and data integration.}
\item { \small Enhance our existing products to make them more robust and scalable while continuously adding new and intuitive features.}

\end{itemize}

\medskip


%----------------------------------------------------------------------------------------
%	PROJECTS SECTION
%----------------------------------------------------------------------------------------



\cvsection[page2sidebar]{\Large Projects}

\cvevent{\large Library System \hspace{0.3em} \href{https://github.com/aseem0510/LibrarySystem}{\textcolor{Lightgr}{\githubsymbol}}}{}{}{}
\begin{itemize}
\item {\small Developed the whole backend pipeline, integrating with PostgreSQL for Library Management System. Tackled various views and routes to obtain a smooth and engaging interface.}
\item {\small Implemented in \textbf{Django} \textbf{Python} using \textbf{PostgreSQL} database.}
\item {\small Fulfilled UI using \textbf{HTML, CSS and JavaScript}.}
\end{itemize}

\medskip


\cvevent{\large Complaint Management System \hspace{0.3em} \href{https://github.com/aseem0510/Complaint-Management-System}{\textcolor{Lightgr}{\githubsymbol}}}{}{}{}
\begin{itemize}
\item {\small Developed a Web-based system to streamline the coordination, monitoring, and resolution of complaints for students and faculties.}
\item {\small Implemented user-friendly features and functionalities to enhance the efficiency of the complaint handling process.}
\item {\smallImplemented Using \textbf{Django}, \textbf{Python} with \textbf{MySQL} database.}
\end{itemize}

\medskip


%----------------------------------------------------------------------------------------
%	CERTIFICATIONS SECTION
%----------------------------------------------------------------------------------------

\cvsection[page2sidebar]{\Large CERTIFICATIONS}
\begin{itemize}
\item {\small Hacker-rank Programmer: \textbf{Problem Solving} and \textbf{Python} Certified.}
% \item {\small Coursera-certified in programming.}
\item {\small \textbf{InfityQ} (by Infosys) Certified Software Engineer.}
\item {\small Coding Ninja Campus Ambassador.}
\end{itemize}



\smallskip

%% If the NEXT page doesn't start with a \cvsection but you'd
%% still like to add a sidebar, then use this command on THIS
%% page to add it. The optional argument lets you pull up the 
%% sidebar a bit so that it looks aligned with the top of the
%% main column.
% \addnextpagesidebar[-1ex]{page3sidebar}



\end{document}
